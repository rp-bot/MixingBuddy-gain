\section{Evaluation}
\label{sec:evaluation}

We evaluate the model configuration described in Section~\ref{sec:training-strategy} to assess its performance on the accuracy of the mixing advice task.
Our model's advice generation is well aligned with the ground truth advice patterns.
This allows us to use the pattern-matching approach to evaluate the accuracy of the generated advice.
We use a test set of 25,650 samples to evaluate the accuracy of the generated advice.
We evaluate the accuracies of the target stem, direction, and the suggested magnitude range in the generated advice.
For each breakdown dimension, we compute both overall performance and performance by error category, stem, and direction.

% \subsubsection{False Positives and False Negatives}

% In addition to accuracy metrics, we analyze false positives and false negatives to understand the types of errors the model makes:

% \begin{itemize}
%     \item \textbf{Error detection}: We track false positives (flagging a problem in an error-free mix) and false negatives (missing an actual mixing error).
%     \item \textbf{Stem identification}: We track when the model selects the wrong stem or fails to detect the correct one.
%     \item \textbf{Direction}: We track when the model predicts the wrong adjustment direction or misses when a direction is needed.
% \end{itemize}

% These error type analyses provide insights into the model's failure modes and help identify whether the model tends to be overly conservative (missing errors) or overly aggressive (flagging non-existent problems), which is crucial for practical deployment in mixing workflows.

% \subsubsection{Performance Breakdowns}

% To gain deeper insights into model behavior, we analyze performance across multiple dimensions:
% \begin{itemize}
%     \item \textbf{By error category}: Performance breakdown across the five error categories (no error, quiet, very quiet, loud, very loud) to identify which types of mixing issues are most challenging.
%     \item \textbf{By target stem}: Performance breakdown across the three target stems (vocals, drums, bass) to assess whether the model has biases toward certain instruments.
%     \item \textbf{By direction type}: Performance breakdown by the required adjustment direction (increase, decrease, no error) to evaluate directional prediction capabilities.
% \end{itemize}

