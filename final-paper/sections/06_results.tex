\section{Results}
\label{sec:results}

On a test set of 25,650 samples, MixingBuddy-gain achieves $49.85\%$ accuracy for the joint task of correctly identifying both the target stem and error direction (e.g., ``The \textbf{drums} are \textbf{too loud}'').
The model achieves 76.06\% accuracy for magnitude range prediction on applicable samples.

Table~\ref{tab:results-fp-fn-f1} details the precision, recall, and F1 scores.
The model demonstrates consistently strong recall across all tasks ($>84\%$), indicating it rarely misses actual mixing problems.
However, precision varies more substantially, with error detection achieving the highest precision, followed by stem identification and direction prediction.
This trade-off is reflected in the F1 scores, where error detection leads, followed by stem identification and direction prediction.

\begin{table}[!t]
\centering
\caption{False positives, false negatives, precision, recall, and F1 scores}
\label{tab:results-fp-fn-f1}
\begin{tabular}{lccccc}
\hline
\textbf{Task} & \textbf{FP} & \textbf{FN} & \textbf{Precision} & \textbf{Recall} & \textbf{F1} \\
\hline
Error Detection & 4079 & 2060 & 81.9\% & 89.96\% & 85.74\% \\
Stem Identification & 5035 & 1545 & 73.47\% & 90.02\% & 80.91\% \\
Direction Prediction & 8028 & 2635 & 64.38\% & 84.63\% & 73.13\% \\
\hline
\end{tabular}
\end{table}


Table~\ref{tab:results-error-category} presents the performance breakdown across the five error categories.
The model shows strongest performance on extreme categories (\textit{very loud} and \textit{very quiet}), suggesting that significant imbalances are more readily detectable.
Magnitude prediction is also particularly reliable for these extreme categories.
Performance drops for moderate categories (\textit{loud} and \textit{quiet}), particularly for magnitude prediction in the \textit{quiet} category.
The model struggles most with the \textit{no error} category, indicating a challenge in correctly identifying well-balanced mixes without false positives.

\begin{table}[!t]
\centering
\caption{Performance breakdown by error category}
\label{tab:results-error-category}
\begin{tabular}{lccccc}
\hline
\textbf{Category} & \textbf{N} & \textbf{Both} & \textbf{Stem} & \textbf{Dir.} & \textbf{Mag.} \\
\hline
very quiet & 5130 & 69.3\% & 72.2\% & 79.9\% & 97.2\% \\
quiet & 5130 & 53.8\% & 62.1\% & 68.8\% & 46.2\% \\
no error & 5130 & 9.3\% & -- & -- & -- \\
loud & 5130 & 38.1\% & 52.8\% & 49.6\% & 63.2\% \\
very loud & 5130 & 78.8\% & 84.7\% & 84.7\% & 97.6\% \\
\hline
\textbf{Macro Avg.} & -- & \textbf{49.9\%} & \textbf{56.2\%} & \textbf{58.4\%} & \textbf{76.1\%} \\
\hline
\end{tabular}
\end{table}

Table~\ref{tab:results-target-stem} shows the performance breakdown across the three target stems.
The model performs best on vocals, followed by drums and bass, for both identification and magnitude prediction.
This suggests that vocal stem identification may benefit from the model's language understanding capabilities, as vocals often contain more semantic content that can be leveraged for identification.

\begin{table}[!t]
\centering
\caption{Performance breakdown by target stem}
\label{tab:results-target-stem}
\begin{tabular}{lccccc}
\hline
\textbf{Stem} & \textbf{N} & \textbf{Both} & \textbf{Stem} & \textbf{Dir.} & \textbf{Mag.} \\
\hline
vocals & 8550 & 55.0\% & 60.3\% & 65.4\% & 93.1\% \\
drums & 8550 & 49.9\% & 54.3\% & 56.2\% & 72.3\% \\
bass & 8550 & 44.7\% & 54.0\% & 53.6\% & 62.8\% \\
\hline
\textbf{Macro Avg.} & -- & \textbf{49.9\%} & \textbf{56.2\%} & \textbf{58.4\%} & \textbf{76.1\%} \\
\hline
\end{tabular}
\end{table}

Table~\ref{tab:results-direction} presents the performance breakdown by adjustment direction.
The model shows slightly stronger performance for \textit{increase} predictions compared to \textit{decrease} predictions for identification.
However, this trend reverses for magnitude prediction, where \textit{decrease} predictions are more accurate.
This suggests that while detecting that a stem is too quiet is easier, determining the precise amount of reduction needed for a loud stem is more reliable once detected.

\begin{table}[!t]
\centering
\caption{Performance breakdown by direction type}
\label{tab:results-direction}
\begin{tabular}{lccccc}
\hline
\textbf{Direction} & \textbf{N} & \textbf{Both} & \textbf{Stem} & \textbf{Dir.} & \textbf{Mag.} \\
\hline
decrease & 10260 & 58.43\% & 68.75\% & 67.12\% & 80.4\% \\
no error & 5130 & 9.3\% & -- & -- & -- \\
increase & 10260 & 61.56\% & 67.12\% & 74.32\% & 71.73\% \\
\hline
\textbf{Macro Avg.} & -- & \textbf{43.09\%} & \textbf{48.38\%} & \textbf{50.24\%} & \textbf{76.06\%} \\
\hline
\end{tabular}
\end{table}

