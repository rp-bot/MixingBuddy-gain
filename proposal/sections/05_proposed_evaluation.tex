\section{Proposed Evaluation}

\subsection{Evaluation Framework}
A comprehensive evaluation strategy will be developed to assess both technical performance and perceptual quality of the automatic mixing system.

\subsection{Objective Metrics}
\begin{itemize}
    \item \textbf{Technical Accuracy}: Comparison of predicted parameters with ground truth values
    \item \textbf{Spectral Analysis}: Frequency response, dynamic range, and harmonic content
    \item \textbf{Level Balancing}: RMS and peak level distributions across tracks
    \item \textbf{Spatial Characteristics}: Stereo width, panning accuracy, and spatial coherence
\end{itemize}

\subsection{Subjective Evaluation}
\begin{itemize}
    \item \textbf{Listening Tests}: A/B comparisons with professional mixes
    \item \textbf{Expert Evaluation}: Assessment by professional audio engineers
    \item \textbf{Genre-Specific Tests}: Evaluation across different musical styles
    \item \textbf{Long-term Listening}: Assessment of mix fatigue and musicality
\end{itemize}

\subsection{Comparative Analysis}
The system will be compared against:
\begin{itemize}
    \item Current state-of-the-art automatic mixing systems
    \item Rule-based mixing approaches
    \item Human-engineered mixes (both amateur and professional)
    \item Baseline systems (e.g., simple level balancing)
\end{itemize}

\subsection{Evaluation Datasets}
\begin{itemize}
    \item \textbf{Professional Mixes}: High-quality reference mixes from various genres
    \item \textbf{Amateur Productions}: User-generated content for generalization testing
    \item \textbf{Synthetic Data}: Generated mixes for controlled evaluation scenarios
    \item \textbf{Cross-Genre Data}: Diverse musical styles for robustness testing
\end{itemize}

\subsection{Statistical Analysis}
\begin{itemize}
    \item Significance testing for subjective evaluations
    \item Correlation analysis between objective and subjective metrics
    \item Error analysis and failure case identification
    \item Performance analysis across different musical contexts
\end{itemize}
