\section{Related Work / Context}

\subsection{Traditional Audio Engineering}
Professional mixing involves complex decision-making processes that balance technical and artistic considerations. Key aspects include:
\begin{itemize}
    \item Level balancing and panning
    \item Equalization (EQ) for frequency shaping
    \item Dynamic processing (compression, limiting)
    \item Spatial effects (reverb, delay)
    \item Harmonic enhancement and saturation
\end{itemize}

\subsection{Existing Automatic Mixing Systems}
\subsubsection{Rule-Based Approaches}
Early automatic mixing systems relied on predefined rules and heuristics \cite{example1}. These systems were limited by their inability to adapt to different musical contexts.

\subsubsection{Machine Learning Approaches}
Recent work has explored various machine learning techniques:
\begin{itemize}
    \item \textbf{Neural Networks}: Deep learning approaches for parameter prediction \cite{example2}
    \item \textbf{Reinforcement Learning}: Systems that learn mixing strategies through trial and error \cite{example3}
    \item \textbf{Generative Models}: Approaches that generate complete mixes from raw tracks \cite{example4}
\end{itemize}

\subsection{Evaluation Methods}
Current evaluation approaches include:
\begin{itemize}
    \item Objective metrics (RMS levels, frequency response, dynamic range)
    \item Subjective listening tests with human evaluators
    \item Comparison with reference mixes
    \item Perceptual models of audio quality
\end{itemize}

\subsection{Gaps in Current Research}
\begin{itemize}
    \item Limited understanding of musical context in mixing decisions
    \item Lack of standardized evaluation frameworks
    \item Insufficient attention to genre-specific mixing requirements
    \item Limited work on interpretable and controllable systems
\end{itemize}
