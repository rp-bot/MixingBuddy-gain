\section{Proposed Method}
\label{sec:proposed_method}

This research introduces and validates a novel, anchor-conditioned framework for fine-tuning an Audio-Language Model (ALM) as a specialized music mixing assistant. 
This approach is designed to emulate the cognitive workflow of human engineers, who often establish a mix foundation by balancing key elements against a stable anchor track (e.g., drums or vocals). 
By conditioning the model on an explicit anchor, we guide it to learn the relational and context-dependent principles of gain balancing. 
The methodology is executed in three distinct phases: anchor-based dataset creation, a multi-stem input formulation, and parameter-efficient fine-tuning.

\subsection{Anchor-Based Dataset for Relational Mixing Instruction}
The foundation of this work is the creation of a new, structured dataset designed specifically to teach the relational reasoning of gain balancing relative to a musical anchor.

\paragraph{Source Audio} We will utilize the high-quality, professionally produced multi-track stems from the MUSDB18 dataset \cite{Rafii_Liutkus_Stöter_Mimilakis_Bittner_2019}. 
The use of this existing, permissively licensed multi-track dataset follows established ethical data sourcing practices.

\paragraph{Data Generation and Annotation} For each song in the MUSDB18 dataset, we will programmatically generate training instances. 
A single instance will consist of a designated anchor track (e.g., the drum stem), a target track (e.g., the vocal stem), and a corresponding text annotation. 
The annotation will be a templated textual instruction describing the ideal gain relationship between the target and the anchor, derived from the professionally mixed reference. 
For example: ``Relative to the drums, the vocal stem is well-balanced. It sits clearly on top of the mix without overpowering the rhythm section.'' 
To create varied training examples, we will also generate versions where the target track is intentionally made too loud or too soft, with corresponding instructional text, e.g., ``Relative to the drums, the vocal stem is too loud. To achieve a better balance, its gain should be decreased.'' 
This strategy provides a scalable method for creating a large, high-quality instruction-following dataset that teaches the model to assess and advise on gain relationships from a musically grounded perspective.

\subsection{Model Architecture and Anchor-Conditioned Input Formulation}
To process the relational audio data, we will employ a state-of-the-art, pre-trained ALM capable of handling multiple audio inputs.

\paragraph{Base Model} We will use Qwen-Audio as the foundational model \cite{Chu_Xu_Zhou_Yang_Zhang_Yan_Zhou_Zhou_2023}. 
Its architecture has demonstrated the ability to process multiple audio streams within a single context, making it exceptionally well-suited for the anchor-based comparison task central to our methodology.

\paragraph{Input Structure} The model will be presented with a targeted set of stems for comparison. 
A typical input will consist of two audio files: the anchor stem (e.g., drums) and a target stem (e.g., vocals). 
These audio files will be accompanied by a text prompt that instructs the model to provide gain-balancing advice for the target stem relative to the anchor stem. 
This focused input structure allows the model to learn the specific gain relationships between pairs of instruments, mirroring a fundamental aspect of the mixing process.

\subsection{Parameter-Efficient Fine-Tuning}
To adapt the pre-trained ALM to our specialized mixing task, we will employ a parameter-efficient fine-tuning (PEFT) methodology.

\paragraph{Methodology} We will use Low-Rank Adaptation (LoRA) to fine-tune the LLM component of the Qwen-Audio model \cite{Hu_Shen_Wallis_Allen-Zhu_Li_Wang_Wang_Chen_2021}. 
LoRA is a standard, computationally efficient method that has been proven highly effective for adapting large models to downstream tasks. 
By inserting small, trainable low-rank matrices into the model's architecture, LoRA allows for significant task-specific adaptation while keeping the vast majority of the base model's parameters frozen. 
This approach dramatically reduces the computational resources required for training and mitigates the risk of catastrophic forgetting.

\paragraph{QLoRA Variant} In resource-constrained settings, we will employ QLoRA, which performs LoRA fine-tuning on a 4-bit quantized copy of the base model \cite{Dettmers_2023_QLoRA}. 
QLoRA preserves model quality while enabling the fine-tuning of large models on a single high-memory GPU, facilitating broader experimentation (e.g., ablations across different anchor-target pairings, prompt formats, and genre contexts) without sacrificing performance.
