\section*{Proposed Method}
\label{sec:proposed_method}

This research presents a framework for fine-tuning an Audio-Language Model (ALM) to generate gain-balancing advice for music mixing. 
The approach conditions the model on a designated anchor track to learn relational understanding of multi-track audio levels.

\subsection*{Data Preprocessing and Dataset Creation}
We will utilize the MUSDB18 dataset \cite{Rafii_Liutkus_Stöter_Mimilakis_Bittner_2019} to create a structured training dataset in JSONL format.

Each training instance will contain a prompt, audio data with anchor track and other track stems, and a response with generated mixing advice describing gain relationships.
Since the original tracks are already well-balanced, we will create flawed versions by artificially adjusting gain levels.

We will create four prompt categories:
\begin{itemize}
    \item \textbf{Audio-only prompts:} Training instances with no text input, requiring the model to analyze audio stems directly.
    
    \item \textbf{Incorrect description prompts:} Training instances where the text prompt contains deliberately wrong descriptions of the audio content (e.g., ``The vocals are too quiet'' when they are actually too loud).
    
    \item \textbf{Advice-seeking prompts:} Training instances with open-ended questions or requests for mixing guidance (e.g., ``How should I balance these tracks?'').
    
    \item \textbf{Correct description prompts:} Training instances with accurate text descriptions of the audio content (e.g., ``The vocal stem is too loud relative to the drums'' or ``The drums are too dominant and overpowering'').
\end{itemize}
We will focus on a single anchor track approach and experiment with scenarios: balanced mixes, one flawed track, and multiple flawed tracks requiring correction.

\subsection*{Model Architecture and Training Strategy}
We will use Qwen-Audio \cite{Chu_Xu_Zhou_Yang_Zhang_Yan_Zhou_Zhou_2023} as the base model, while exploring other pre-trained Audio-Language Models for fine-tuning.
\begin{figure*}[htbp]
    \centering
    \includegraphics[width=\textwidth]{figures/arch.pdf}
    \caption{Proposed architecture of Audio-Language Model fine-tuning for music mixing advice generation.}
    \label{fig:architecture}
\end{figure*}
We will use Low-Rank Adaptation (LoRA) \cite{Hu_Shen_Wallis_Allen-Zhu_Li_Wang_Wang_Chen_2021} and QLoRA \cite{Dettmers_2023_QLoRA} for parameter-efficient fine-tuning.

\subsection*{Data Augmentation and Cross-Sample Training}
We will implement cross-sample training strategies to improve generalization.
We will explore creating hybrid training examples by combining stems from different songs while maintaining musical coherence.

% \subsection*{Architecture Experiments}
% We will conduct systematic experiments to optimize the model architecture and training approach.
% We will evaluate different anchor selection strategies and their impact on model performance across various musical contexts and genres.
