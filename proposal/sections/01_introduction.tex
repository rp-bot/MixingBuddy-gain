\section*{Introduction}

Current Audio-Language Models (ALMs) show promise for descriptive audio tasks, but effective music mixing requires a relational understanding of multi-track audio that goes beyond simple analysis. 
This research investigates a novel framework where an ALM is conditioned on a designated anchor track to learn the relative levels of other stems and generate actionable gain-balancing advice. 
The study will leverage the MUSDB18 dataset for training and evaluation, combining automated metrics with qualitative assessments from audio engineers to gauge the musicality, effectiveness, and real-world usefulness of the generated advice.

\subsection*{Research Questions}

\subsubsection*{Primary Research Question}
To what extent can an Audio Language Model, conditioned on an anchor track, learn about relative levels of multi-track audio to generate effective gain-balancing advice for music mixing?

\subsubsection*{Secondary Research Questions}
\begin{enumerate}
    \item \textbf{Architecture} What is an effective model architecture for representing multi-track stems and anchor tracks, enabling an Audio Language Model to learn and reason about their relative levels for music mixing?
    \item \textbf{Learned Conventions} Does the model's generated advice demonstrate an understanding of established mixing conventions and genre-specific expectations (e.g., the typical vocal-to-instrument balance in pop versus jazz)?
    \item \textbf{Evaluation \& Usefulness} How do audio engineers and producers rate the effectiveness, musicality, and actionability of the generated gain-balancing advice, and what qualitative feedback do they provide on its integration into their workflow?
    \item \textbf{Metric Correlation} What is the correlation between subjective human preference judgments and automated evaluation metrics (e.g., LLM-as-a-Judge, BERTscore) for evaluating mixing advice?
\end{enumerate}

\subsection*{Scope and Limitations}
The model's output will be advisory text that may include gain predictions, but evaluation will focus on the overall quality and usefulness of the advice rather than the accuracy of specific numerical predictions. 
The investigation excludes other mixing parameters such as equalization (EQ), dynamic range compression, and spatial effects. 
The proposed system is designed for offline analysis and is not intended for real-time, interactive applications. 
The validity of this work relies on the assumption that the professionally mixed versions of the tracks in the MUSDB18 dataset represent a perceptually valid ground truth for a well-balanced mix and that the dataset is of sufficient quality and diversity for the task.
