\section*{Related Work}
\label{sec:related_work}

The field of ``Co-Creative AI'' provides a compelling framework for music production, envisioning AI not as a replacement for human creativity, but as an intelligent partner that augments the artistic workflow \cite{Louie_Coenen_Huang_Terry_Cai_2020b}. 

\subsection*{Audio Mixing Paradigms}
Traditional automated mixing approaches have evolved through several paradigms:
Rule-based methods employed expert knowledge systems \cite{Dugan_1976, Perez-Gonzalez_Reiss_2013, Mansbridge_Finn_Reiss_2012}, while recent black box systems take raw multitrack audio and produce fully mixed outputs with minimal user intervention \cite{Steinmetz_Pons_Pascual_Serra_2020, Martinez-Ramirez_Liao_Fabbro_Uhlich_Nagashima_Mitsufuji_2022}. 
However, these approaches limit the fine-grained control required in professional workflows.

In contrast, the emerging Language Bridge paradigm uses natural language as the primary interface for interaction, allowing users to articulate creative intent in descriptive terms.
Recent work has begun exploring this direction \cite{Clemens_Marasović_2025, Chu_OReilly_Barnett_Pardo_2025, Doh_Koo_Martinez-Ramirez_Liao_Nam_Mitsufuji_2025, Melechovsky_Mehrish_Herremans_2025, Lai_Hung_Zhu_Wang_Sheu_Juang_2022, Pardo_Cartwright_Seetharaman_Kim_2019, Venkatesh_Moffat_Miranda_2022}.

\subsection*{Audio-Language Models}
The development of audio-language models has progressed from task-specific supervised models to more sophisticated approaches.
Contrastive dual-encoder models like CLAP learn audio concepts from natural language supervision, enabling zero-shot audio classification and retrieval \cite{Elizalde_Deshmukh_Ismail_Wang_2023, Koepke_Oncescu_Henriques_Akata_Albanie_2023}.
Large Audio-Language Models (LALMs) represent a significant advancement, with models demonstrating language modeling approaches to audio generation \cite{Borsos_Marinier_Vincent_Kharitonov_Pietquin_Sharifi_Roblek_Teboul_Grangier_Tagliasacchi_etal_2023, zhao2023survey}.

Multimodal Large Language Models have emerged with different architectural approaches, including modality interface architectures that augment pre-trained LLMs with specialized audio encoders \cite{Chu_Xu_Zhou_Yang_Zhang_Yan_Zhou_Zhou_2023, Tang_Yu_Sun_Chen_Tan_Li_Lu_Ma_Zhang_2024}, and any-to-any systems that enable both perception and generation across multiple modalities \cite{Vyas_Shi_Le_Tjandra_Wu_Guo_Zhang_Zhang_Adkins_Ngan_eta_2023, Wu_Fei_Qu_Ji_Chua_2024}.

\subsection*{LLM Integration Approaches}
The integration of LLMs in audio tasks follows several key patterns \cite{Yin_Fu_Zhao_Li_Sun_Xu_Chen_2024}:
\begin{itemize}
    \item LLMs as Backbone: Pre-trained LLMs serve as the central architecture with modality-specific encoders/decoders
    \item LLMs as Conditioner: LLMs encode text prompts into embeddings that condition audio generation
    \item LLMs as Agent: LLMs act as controllers, coordinating external tools to accomplish audio tasks
\end{itemize}

Our research builds upon this foundation, specifically targeting the Language Bridge paradigm for audio mixing applications.
