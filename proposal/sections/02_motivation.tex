\section{Motivation}

\subsection{Industry Need}
The music production industry faces increasing pressure to deliver high-quality mixes efficiently. Professional mixing engineers spend significant time on repetitive tasks that could potentially be automated, while amateur producers often lack the expertise to achieve professional-quality results.

\subsection{Technical Challenges}
Current automatic mixing systems suffer from several limitations:
\begin{itemize}
    \item \textbf{Lack of Context Awareness}: Most systems process tracks independently without considering the musical context or genre-specific requirements
    \item \textbf{Limited Musical Understanding}: Existing approaches often focus on technical audio features without understanding musical structure and relationships
    \item \textbf{Evaluation Difficulties}: There is no standardized way to evaluate mixing quality that correlates well with human perception
    \item \textbf{Generalization Issues}: Systems trained on specific datasets often fail to generalize to different musical styles or recording conditions
\end{itemize}

\subsection{Research Impact}
Successful development of advanced automatic mixing systems could:
\begin{itemize}
    \item Democratize access to professional-quality music production
    \item Reduce production costs and time-to-market for music releases
    \item Enable new creative workflows and collaborative approaches
    \item Advance the field of computational audio processing
\end{itemize}
