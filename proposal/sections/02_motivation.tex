\section{Motivation}

\subsection{Research Gap}
A significant gap exists in the current automatic mixing research landscape. 
The field is largely dominated by black-box, deep learning models that, while achieving impressive results, offer little to no explainability into their decision-making process. 
This contrasts with other creative domains like 3D modeling and image editing, which have seen the rise of end-to-end agentic systems that facilitate user interaction and provide transparency.

Furthermore, while generative audio models have gained popularity, their output is often of limited value to professional mixing and mastering engineers, who require precise control over mixing parameters rather than a final, uneditable audio file.

To bridge this gap, future research must move towards models with better audio and contextual understanding. Exploring techniques like audio-conditioned chain-of-thought reasoning, combined with state-of-the-art automatic mixing models, will be crucial. This direction will not only improve performance but also foster trust and adoption by professionals by providing explainable and controllable systems.

\subsection{Research Impact}
The development of explainable and interactive automatic mixing systems could have a profound impact on the music production industry. Digital Audio Workstations (DAWs) could integrate features where users can converse with an intelligent mixing agent to receive advice, ask specific questions, and collaboratively edit mixes. This paradigm shifts the focus from replacing engineers with end-to-end generative systems to empowering musicians and producers with powerful, intuitive tools. By fostering a more interactive and transparent mixing process, this research can enhance creativity and streamline workflows for audio professionals and hobbyists alike.
