\section{Proposed Method}

\begin{frame}\frametitle{\secname}\framesubtitle{Overview}
    \begin{columns}[T,totalwidth=\textwidth]
        \column{0.6\textwidth}
        \begin{itemize}
            \item<1-> \textbf{LLM as backbone}: A pretrained LLM such as Qwen2 \cite{yang2024qwen2technicalreport} as the backbone.
            \item<2-> \textbf{Input}: A Flawed mix.
            \item<3-> \textbf{Output}: A structured response containing advice pointing out the flaws and suggesting solutions.
            \item<4-> \textbf{Architecture}: Cascade approach, with the LLM as the backbone.
            \item<5-> \textbf{Training strategy}: Supervised fine-tuning using PEFT (Parameter-Efficient Fine-Tuning), specifically LoRA \cite{Hu_Shen_Wallis_Allen-Zhu_Li_Wang_Wang_Chen_2021} or QLoRA \cite{Dettmers_2023_QLoRA}.
        \end{itemize}

        \column{0.4\textwidth}
        \centering
        % Overview architecture variants
        \only<1>{\includegraphics[height=0.8\textheight,keepaspectratio]{pictures/arch-1.png}}
        \only<2>{\includegraphics[height=0.8\textheight,keepaspectratio]{pictures/arch-2.png}}
        \only<3>{\includegraphics[height=0.8\textheight,keepaspectratio]{pictures/arch-3.png}}
        \only<4->{\includegraphics[height=0.8\textheight,keepaspectratio]{pictures/arch-4.png}}
    \end{columns}
\end{frame}

\begin{frame}\frametitle{\secname}\framesubtitle{Dataset Synthesis (The Response)}
    \begin{columns}[T,totalwidth=\textwidth]
        \column{0.5\textwidth}
        \begin{itemize}
            \item<1-3> \textbf{Overview}: Our training data is synthetically generated by formulating diverse input-response pairs.
            \item<2-> \textbf{Flaw-Driven Templating}: A ``Flaw Category'' is randomly selected, dictating the structure and content of the response.
            \item<3-3> \textbf{Dynamic Population}: Template variables such as \textbf{stem names} and \textbf{suggested gain values} are automatically populated.
            % \item<4-> \textbf{Output Pairs}: The synthesis process yields comprehensive (Implicit Instruction, Ideal Response) triplets for model training.
        \end{itemize}

        \column{0.5\textwidth}
        \only<1-3>{
            \centering
            \includegraphics[height=0.8\textheight,keepaspectratio]{pictures/response-synthesis.png}
        }
        
        \only<4>{
            \textbf{Key Mixing Flaw Categories for Synthesis:}
                \begin{itemize}
                    \item \textbf{No Error}:  \textit{``The mix sounds balanced.''}
                    \item \textbf{Quiet}: \textit{``The vocal is too quiet.''}
                    \item \textbf{Very Quiet}: \textit{``The vocal is much too quiet.''}
                    \item \textbf{Loud}: \textit{``The bass is too loud.''}
                    \item \textbf{Very Loud}: \textit{``The bass is much too loud.''}
                \end{itemize}
        }
        
        
    \end{columns}

\end{frame}

\begin{frame}\frametitle{\secname}\framesubtitle{Dataset Synthesis (The Flawed Mix)}
    \begin{columns}[T,totalwidth=\textwidth]
        \column{0.5\textwidth}
        \begin{itemize}
            \item<1-> \textbf{Dataset}: A multitrack dataset like MUSDB18 \cite{Rafii_Liutkus_Stoter_Mimilakis_Bittner_2019}.
            \item<2-> Chunk a song into 10-second segments.
            \item<3-> Inject an error of $\pm n$ dB on a non-anchor track based on Flaw Categories.
            \item<4-> Sum the stems to get the flawed mix.
        \end{itemize}

        \column{0.5\textwidth}
        \centering
        \only<1>{\includegraphics[height=0.8\textheight,keepaspectratio]{pictures/mix-synthesis-1.png}}
        \only<2>{\includegraphics[height=0.8\textheight,keepaspectratio]{pictures/mix-synthesis-2.png}}
        \only<3>{\includegraphics[height=0.8\textheight,keepaspectratio]{pictures/mix-synthesis-3.png}}
        \only<4>{\includegraphics[height=0.8\textheight,keepaspectratio]{pictures/mix-synthesis-4.png}}
    \end{columns}

\end{frame}



