\section{Proposed Method}

\begin{frame}\frametitle{\secname}\framesubtitle{Overview}
    \begin{columns}[T,totalwidth=\textwidth]
        \column{0.6\textwidth}
        \begin{itemize}
            \item<1-> \textbf{LLM as backbone}: A pretrained Audio LLM like Qwen-Audio \cite{Chu_Xu_Zhou_Yang_Zhang_Yan_Zhou_Zhou_2023} as the backbone.
            \item<2-> \textbf{Input}: Anchor track, rest of the tracks, and a text prompt.
            \item<3-> \textbf{Output}: A structured response containing \textbf{advice} for the user to \textbf{balance the levels} of the tracks.
            \item<4-> \textbf{Architecture}: Cascade approach, with the LLM as the backbone.
            \item<5-> \textbf{Training strategy}: Supervised fine-tuning using PEFT (Parameter-Efficient Fine-Tuning), specifically LoRA \cite{Hu_Shen_Wallis_Allen-Zhu_Li_Wang_Wang_Chen_2021} or QLoRA \cite{Dettmers_2023_QLoRA}.
        \end{itemize}

        \column{0.4\textwidth}
        \centering
        % Overview architecture variants
        \only<1>{\includegraphics[height=0.8\textheight,keepaspectratio]{pictures/arch1.png}}
        \only<2>{\includegraphics[height=0.8\textheight,keepaspectratio]{pictures/arch2.png}}
        \only<3>{\includegraphics[height=0.8\textheight,keepaspectratio]{pictures/arch3.png}}
        \only<4->{\includegraphics[height=0.8\textheight,keepaspectratio]{pictures/arch4.png}}
    \end{columns}
\end{frame}

\begin{frame}\frametitle{\secname}\framesubtitle{Dataset Synthesis}
    \begin{itemize}
        \item<1-> \textbf{Prompt}: Create per-sample prompts in addtion to the constant system prompt.
        \item<2-> \textbf{Multitrack Input}: 
        \begin{itemize}
            \item<2-> \textbf{Dataset}: A multitrack dataset like MUSDB18 \cite{Rafii_Liutkus_Stoter_Mimilakis_Bittner_2019}.
            \item<3-> Chunk a song into 10-second segments.
            \item<4-> Inject an error of $\pm n$ dB on a non-anchor track.
        \end{itemize}
        \item<5-> \textbf{Response Formulation}: Programmitically create structured responses.
        \begin{itemize}
            \item<5-> Create templates based on the prompt. The templates will be informed by established mixing conventions.
            \item<6-> Pick a template based on the prompt, containing both description and solution placeholders.
            \item<7-> Populate the placeholders with the ground truth description and solution.
        \end{itemize}
    \end{itemize}

\end{frame}



